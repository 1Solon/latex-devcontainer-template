Public authorities, as a core requirement, must choose where to place services.
This requires a comprehensive understanding of the availability of other
services, such as parking in the area. Traditional methods of gathering this
information can be time-consuming and inefficient, often involving consulting
maps and other complex data. Therefore, a solution is needed that can
efficiently analyze this information. This solution could leverage the growing
field of geospatial data analysis, specifically the integration of machine
learning with satellite imagery. This approach can provide a more accurate and
real-time understanding of local services, facilitating more effective planning
and management of public services.\\

\noindent{}Literature in the field of Computer Vision for geospatial
applications highlights significant advancements in object recognition and image
analysis. Research by X and Y demonstrates the efficacy of convolutional neural
networks (CNNs) in detecting physical objects from aerial images, which aligns
closely with the objectives of this project. Similarly, studies on data fusion
techniques, as discussed by Z, provide a foundation for integrating aerial
imagery with other data sources, such as municipal records of underground
parking and public transportation facilities. (TODO: Re-add refrences)\\

\newpage{}

\noindent{}Situated within the broader computer science domain, this project
contributes to the fields of geospatial analysis, data integration, and
application development. (TODO: IMPROVE THIS) It addresses practical needs and
pushes the envelope in applying machine learning techniques in real-world
scenarios.\\

\noindent{}Subsequent phases will expand the application’s capabilities to
include data integration, whereby information on other logistical elements such
as underground parking availability and public transport links will be
introduced as metrics. This approach aims to provide a comprehensive tool for
urban logistics, as much data related to the domain as possible should
be colated.\\

\newpage{}

\section{Approaches to Solving the Problem}
\subsection{Machine vision smart parking using internet of things (IoTs) in a smart university}
\cite{sieck2020machine}

\subsection{Research on Parking Space Status Recognition Method Based on Computer Vision}
\cite{li2022research}

\subsection{Deep Learning based Automated Parking Lot Space Detection using Aerial Imagery}
\cite{gopinath2023deep}

\subsection{A Web Application Exhibiting Parking Guidance using Smart Sensor Networks}
\cite{sivakumar2020web}

\subsection{A systematic review of machine-vision-based smart parking systems}
\cite{abidin2020systematic}

\subsection{An adaptive vision-based outdoor car parking lot monitoring system}
\cite{nguyen2021adaptive}

\subsection{Smart Vehicle Parking System Using Computer Vision and Internet of Things (IoT)}
\cite{taylor2021smart}

\subsection{Autonomous Parking-Lots Detection with Multi-Sensor Data Fusion Using Machine Deep Learning Techniques.}
\cite{iqbal2021autonomous}

\subsection{Automatic vision-based parking slot detection and occupancy classification}
\cite{grbic2023automatic}